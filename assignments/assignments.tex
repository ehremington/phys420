\hypertarget{problem-set-1}{%
\section{Problem Set 1}\label{problem-set-1}}

\hypertarget{problem-1.1}{%
\subsection{Problem 1.1}\label{problem-1.1}}

\hypertarget{problem-1.3}{%
\subsection{Problem 1.3}\label{problem-1.3}}

\hypertarget{problem-1.12}{%
\subsection{Problem 1.12}\label{problem-1.12}}

\hypertarget{proportionality}{%
\subsection{Proportionality}\label{proportionality}}

Proportionality is a very convenient way to summarize a relationship,
but its the kind of thing that everyone assumes someone else taught you.

So when we see an equation like the ideal gas law \(P V = N k_B T\) this
equation contains many relationships that can be stated as
\emph{ratios}. As an example, imagine we are conducting an experiment
where we vary the temperature of a gas, but we do not change the number
of particles or the volume of the container. The question is what
happens to the pressure? We can relate the \emph{ratio} of the two
temperatures to the \emph{ratio} of the two different pressures. Here is
how:

Solve for the Pressure:

\[ P = \frac{N k_B}{V} T \]

and then see that this general equation must be satisfied in the
specific instances of \(T_1, T_2, P_1, P_2\):
\[ P_1 = \frac{N k_B}{V} T_1 \] \[ P_2 = \frac{N k_B}{V} T_2 \]

Dividing the two equations as 2/1, shows that all of the constant things
(\(N, k_b, V\)) cancel out and we are left with a proportion

\[\frac{P_2}{P_1} = \frac{T_2}{T_1}\]

This type of proportionality is called \emph{directly proportional} or
\emph{linearly proportional} since if you double the temperature
\(T_2/T_1 = 2\) then the equation says that the pressure would double as
well \(P_2/P_1 = 2\).

The fancy/mathy way of writing this is \(P \propto T\). When you see
that statement, it means you can write down
\[\frac{P_2}{P_1} = \frac{T_2}{T_1}\]

Other types of proportionality follow from this logic. So for example

\[ y \propto \frac{1}{x} \]

is called \emph{inverse proportionality}. It is also sometimes written
as \(y \propto x^{-1}\). From these statements, you can write down

\[ \frac{y_2}{y_1} = \frac{x_1}{x_2} = \left(\frac{x_2}{x_1}\right)^{-1} \]

There are other proportionality statements that are very important in
physics like the famous \emph{inverse square law} which comes in the
form \(y \propto x^{-2}\). These kinds of proportionality statements can
also be written as an equation with a \emph{constant of proportionality}
which is often the mores familiar form to us. So for example the law of
Universal Gravitation can be written like

\[ F = \frac{G m_1 m_2}{r^2} \]

We could say the the force is directly proportional to each mass, and
inversely proportional to the square of the distance between them. The
constant of proportionality is \emph{G}.

Ok, now go back to the ideal gas law and imagine the following
scenarios:

\begin{enumerate}
\def\labelenumi{\arabic{enumi}.}
\tightlist
\item
  By what factor does the pressure change if the temperature triples
  (here we assume that all other things are unchanged)? (that language
  of ``by what factor'' is just another way of saying ``whats the ratio
  of pressures'')
\item
  By what factors does the pressure change if the number of particles
  doubles? What about halving? (remember that this is only true if the
  temperature is the same as well as the volume - how would you do
  that?).
\item
  By what factor does the pressure change if the volume doubles?
\item
  By what factor does the pressure change if the volume doubles and the
  temperature triples?
\item
  By what factors would the temperature change if the volume doubled and
  the pressure tripled?
\item
  By what factors would the volume change if the number of particles
  halved and the pressure doubled and the temperature \emph{decreased}
  by a factor of 10? How would you do this in the lab?
\end{enumerate}
